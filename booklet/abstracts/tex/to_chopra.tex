
    \begin{abstract_online}{Learning to Simulate Millions of Agents by bridging AI and Agent-based Models}{%
        A. Chopra}{%
        }{%
        Massachusetts Institute of Technology, Boston, MA}
    Humanity is facing grand challenges at unprecedented rates, nearly everywhere, and at all levels. Many of these challenges: pandemics, financial market instability and disinformation in social media, are emergent phenomena that result from complex interactions between a large number of strategic agents. Agent-based modeling (ABM) helps simulate such complex systems by modeling the actions and interactions of individual agents contained within. In recent efforts to contain the COVID-19 pandemic, ABMs have been used to decide lockdown strategies and prioritize vaccination schedules. In financial markets, ABMs can be used to understand effects of clustered volatility, regulatory changes and systemic risk. The utility of ABMs for practical decision making depends upon their ability to recreate populations with great detail, efficiently calibrate to real-world data and analyze sensitivity of results. However, ABMs are conventionally slow to execute, difficult to scale to large populations and tough to calibrate. My research aims to alleviate these challenges by fundamentally rethinking ABMs in this era of AI. My research introduces GradABM: a scalable, differentiable ABM design that is amenable to gradient-based learning with automatic differentiation - and hence can benefit from recent advances in AI. GradABMs can simulate million-size populations in a few seconds on commodity hardware, integrate with deep neural networks and ingest heterogeneous data sources. My work, over the past year, has demonstrated benefits of GradABM for scalable and fast simulations on both CPUs and GPUs, data-driven and robust calibration (by coupling with neural networks) and with uncertainty quantification (via generative neural modeling) as well as efficient validation using gradient-based sensitivity analyses. On a popular epidemiological model to simulate 60 million agents in the UK, (re-)designing as a GradABM reduced simulation time from 50 hours to 5 minutes, calibration time from 10,000 hours to 20 minutes and sensitivity analysis time from 5,000 hours to 10 seconds. In collaboration with Mayo Clinic and UN Global Pulse, this GradABM has been used to conduct digital experimentation at real world scale by evaluating prospective policy interventions (validate immunization protocols) as well as analyzing retrospective decisions (reproduce city-scale seroprevalence studies in-silico). More recently, we have introduced AgentTorch, a low-code framework for building custom GradABMs across digital, physical and biological realms. This will enable domain experts - epidemiologists, immunologists, economists etc - to rapidly iterate and benefit from the capabilities of AI and GradABMs, while abstracted from the engineering complexity.  
    
    \end{abstract_online}
    