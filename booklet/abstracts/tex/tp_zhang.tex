
        \begin{abstract}{Heterogeneous changes in mobility in response to the SARS-CoV-2 Omicron BA.2 outbreak in Shanghai}{%
            J. Zhang}{%
            Fudan University}{%
            }
        The coronavirus disease 2019 (COVID-19) pandemic and the measures taken by authorities to control its spread had altered human behavior and mobility patterns in an unprecedented way. However, it remains unclear whether the population response to a COVID-19 outbreak varies within a city or among demographic groups. Here we utilized passively recorded cellular signaling data at a spatial resolution of 1km x 1km for over 5 million users and epidemiological surveillance data collected during the SARS-CoV-2 Omicron BA.2 outbreak from February to June 2022 in Shanghai, China, to investigate the heterogeneous response of different segments of the population at the within-city level and examine its relationship with the actual risk of infection. Changes in behavior were spatially heterogenous within the city and population groups, and associated with both the infection incidence and adopted interventions. We also found that males and individuals aged 30-59 years old traveled more frequently, traveled longer distances, and their communities were more connected; the same groups were also associated with the highest SARS-CoV-2 incidence. Our results highlight the heterogeneous behavioral change of the Shanghai population to the SARS-CoV-2 Omicron BA.2 outbreak and the its effect on the heterogenous spread of COVID-19, both spatially and demographically. These findings could be instrumental for the design of targeted interventions for the control and mitigation of future outbreaks of COVID-19 and, more broadly, of respiratory pathogens. 
        \end{abstract}
        