
        \begin{abstract}{None}{%
            S. Yitbarek}{%
            UNC Chapel Hill, NC}{%
            }
        The distribution of mosquitoes and associated vector diseases (e.g., West Nile, dengue, and Zika viruses) is likely a function of environmental conditions in the landscape. Urban environments are highly heterogeneous in the amount of vegetation, standing water, and concrete structures covering the land at a given time, each having the capacity to influence mosquito abundance and disease transmission. Here, we present a meta-analysis of 42 paired observations from 18 articles testing how socioeconomic status relates to overall mosquito burden in urban landscapes in the US. We also analyzed how with socio-ecological covariates (e.g., abandoned buildings, vegetation, education, and garbage containers) varied across socioeconomic status in the same mosquito studies. The meta-analysis revealed that lower-income neighborhoods (regions with median household incomes $<$\$50,000 household-1 year-1) are exposed to 63\% greater mosquito densities and mosquito-borne illnesses compared to higher-income neighborhoods ($\geq$\$50,000 household-1 year-1). One common species of urban mosquito (Aedes aegypti) showed the strongest relationship with socioeconomic status, with Ae. aegypti being 126\% higher in low-income than high-income neighborhoods. We also found that certain socio-ecological covariates correlated with median household income. Garbage, trash, and plastic containers were found 67\% higher in low-income neighborhoods, whereas high-income neighborhoods tended to have higher levels of education. Together, these results indicate that socio-ecological factors can lead to disproportionate impacts of mosquitoes on humans in urban landscapes. Thus, concerted efforts to manage mosquito populations in low-income urban neighborhoods are required to reduce mosquito burden for the communities most vulnerable to human disease. 
        \end{abstract}
        