
    \begin{abstract_online}{Measles Transmission in household structured models}{%
        S. Bidari}{%
        }{%
        Columbia University, New York, NY}
    Households are known to have a significant impact on the transmission of many diseases, yet its impact has not been fully accounted for in commonly used compartmental models. Many studies on models of disease transmission with household level mixing consider a static household distribution. These approaches have provided valuable insights on final epidemic sizes, threshold parameters, and vaccination policies but fail to consider the demographic changes in household structure for endemic diseases like measles. To capture the long-term evolution of households and dynamics of disease transmission within them, we introduce a model of disease transmission that includes demographic changes in the household sizes. We explore the variation in transmission dynamics caused by the differential changes in contact networks and population demographic over several decades of measles persistence in China. Using model simulations, we show that incorporating only age structured mixing systematically overestimate the infection compared to the models with both household and age structured mixing. Our model provides a comprehensive framework to understand the spread and endemicity of measles by incorporating household level mixing in populations without detailed household level data. 
    
    \end{abstract_online}
    