
    \begin{abstract_online}{Competing Social Contagions with Opinion Dependent Infectivity}{%
        C. Sampson}{%
        }{%
        University of Colorado Boulder, Boulder, CO}
    The rise of online social media as a form of communication and a source of news has allowed for the rapid dissemination of information within social networks. However, this has also allowed for the easy spread of false information by malicious users and social-bot accounts. The spread of information within a network of interacting individuals is often modeled as a contact process where information is spread between two individuals in a manner analogous to the spread of an infectious disease. In this context the spread of information is referred to as a social contagion. The spread of a social contagion may be affected by social phenomena such as selective exposure bias and the mere exposure effect. We are interested in the spread of misinformation, particularly if a piece of false information, for example a rumor, can out-compete the spread of a mutually exclusive “truth” based only on the formation of social biases through these mechanisms. Here we consider a model of social contagion spreading as a contact process on a network where individuals have both a trinary “infection” state, which represents if the individual is an active spreader of a rumor or the corresponding “truth” about that rumor, and an opinion state, which represents the individual’s belief in either the rumor or truth. Furthermore, the infection and recovery rates are functions of the individual’s opinion, which is influenced by exposures to “infected” individuals. This model results in a set of mean-field equations which are reminiscent of the susceptible-infectious-susceptible (SIS) model of epidemiology. The epidemic threshold depends on the average opinion of the population, which results in an interesting rebound behavior in which the fraction of infected individuals initially decreases, is maintained by a small fraction of the populations, and then rebounds to an endemic state. In addition, the competition between the two mutually exclusive social contagions can result in overturning the opinion of an initially biased population towards the state opposite of the bias. This can occur with only a slight increase in the initial number of infected individuals in the state opposite of the population’s bias, suggesting that an initial opinion bias within a population may be vulnerable to mass spamming of misinformation. 
    
    \end{abstract_online}
    