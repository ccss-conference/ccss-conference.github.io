
    \begin{abstract_online}{None}{%
        A. Colubri}{%
        }{%
        University of Massachusetts Chan Medical School, Boston, MA}
    We will introduce Operation Outbreak (OO), an app-based platform for experiential learning simulations of infectious disease outbreaks in real-life settings, and will discuss OO applications in network epidemiology and behavioral research. During an OO simulation, a virtual pathogen spreads through nearby phones using Bluetooth. A realistic epidemiological model drives the pathogen and can be customized to represent different scenarios. The app informs the participants of their simulated health status. They can optionally use QR-code items such as masks and vaccines to protect themselves. The app keeps track of the full history of events in the simulation (contacts, infections, recovery, etc.) and stores this data in the backend in real-time, enabling subsequent analysis and visualizations. Participants in school settings can play a variety of roles representing groups and institutions in society (e.g., general population, government, epidemiologists) while responding to an outbreak and make cost-benefit decisions affected by in-game rewards. We have repeatedly observed how socio-behavioral parallels between our simulated outbreaks and real epidemics recapitulated observations from past studies on participants’ engagement and perceptions of authenticity of the experience. Recent work published by our team using data from large-scale OO simulations at college has also shown the potential of OO as a source of real-time data for modeling by revealing cryptic transmission paths and the impact of various control measures, as the data from these simulations can be used to characterize the contact networks of participants. We will propose new gamification features in the OO platform to enable longer-running simulations suited for general audiences, with the purpose of generating outbreak datasets that capture behavioral patterns of large number of participants through their contact traces, and how their decisions during the simulation affect the spread of the virtual pathogen and mirror nonpharmaceutical interventions in real-life outbreaks. As an example, participants could “quarantine” by selecting a button in the OO app. They would earn points for every day they are not in this virtual quarantine, representing the individual-level benefits of being able to go to work, socialize, etc. Participants would lose points if they became infected and would be informed of their individualized cost at the beginning of the game. This should induce variation in individual’s attitudes around contact tracing and quarantine because compliance with measures will be more important for individuals with higher infection costs. We will place this work within the framework of network epidemiology and how the OO data would be able to inform network-based forecasting models. Finally, we propose running a live OO simulation during the CCSS 2023 conference and use the OO tools so attendants can visualize the dynamics of the virtual outbreak in real-time. 
    
    \end{abstract_online}
    