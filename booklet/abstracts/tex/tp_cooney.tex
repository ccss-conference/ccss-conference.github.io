
        \begin{abstract}{Social Dilemmas of Sociality due to Beneficial and Costly Contagion}{%
            D. Cooney}{%
            UPenn, MA}{%
            }
        Levels of sociality in nature vary widely from solitary species to complex multi-family societies. Increased levels of social interaction can allow for the spread of useful innovations and beneficial information, but can also facilitate the spread of harmful contagions, such as infectious diseases. In this talk, we will explore how coupled contagion processes can help shape the rules for interaction in complex social systems. We consider a model for the evolution of sociality strategies in the presence of both a beneficial and costly contagion, and study dynamics of this model at multiple timescales. We use a susceptible-infectious-susceptible (SIS) model to describe contagion spread for given sociality strategies, and then employ the adaptive dynamics framework to study the long-time evolution of the levels of sociality in the population. For a wide range of assumptions about the benefits and costs of infection, we identify a social dilemma: the evolutionarily-stable sociality strategy (ESS) produced by adaptive dynamics is distinct from the collective optimum -- the level of sociality that would be best for all individuals. In particular, the ESS level of social interaction is greater (respectively less) than the social optimum when the good contagion spreads more (respectively less) readily than the bad contagion. Our results shed light on how contagion shapes the evolution of social interaction, but reveals that evolution may not necessarily lead populations to social structures that are good for any or all. This project is joint work with Dylan H. Morris, Simon A. Levin, Daniel I. Rubenstein, and Pawel Romanczuk. 
        \end{abstract}
        