
    \begin{abstract_online}{Emerging call for action: The complex and paradoxical co-evolution of contagions and institutions}{%
        J. St-Onge}{%
        }{%
        University of Vermont, VT}
    Epidemic models are used to study the spread of an undesired agent through a population, be it diseases infecting countries, misinformation adulterating social media, or pests blighting regions. In fighting these epidemics, we do not exclusively depend on either global top-down interventions or individual adaptations. Interventions often come from local institutions such as public health departments, moderation teams on social media, or other forms of group governance. We leverage recent development of institutional dynamics to investigate the intermediary scale of groups, which is understudied compared to macro-level top-down interventions or micro-scale adaptive individual behaviour. Using principles adapted from group selection theory, we model meso-scale groups attempting local control of an epidemic through adaptation based on successes and failures of other groups. This modeling approach results in a hypergraph model where institutions can emerge and grow on hyperedges (representing groups) to locally affect the epidemic dynamics. In this model, we find complex co-evolutionary dynamics which we summarize as five possible dynamical regimes (see figure). Across all regimes, we find that a faster rate of policy imitation leads to a higher steady-state prevalence. Fast imitation is beneficial in the early phase, but high reactivity does not give enough time for stronger policies to prove their efficacy, eventually leading to an abundance of weak institutions unable to control the epidemic. Additionally, the initial conditions determine the transient behaviors. In particular, if groups with stronger policies are present from the outset, the magnitude of epidemic waves is greatly reduced and slightly delayed in time. Altogether our results illustrate the complex dynamics missed by models that ignore the dynamical interplay of contagions with group interventions. 
    
    \end{abstract_online}
    