
    \begin{abstract_online}{The statistical and dynamic modeling of protests in Ukraine: the revolution of dignity and preceding times}{%
        Y. Bahid}{%
        }{%
        University of Colorado Boulder, Boulder, CO}
    Ukraine's tug-of-war between Russia and the West has had significant and lasting consequences for the country.  In 2013 the Russian-leaning Ukrainian president, Viktor Yanukovych,  refused to sign the association agreement with the European Union. This led to widespread protests centered in Kyiv's Maidan Square.  The protests went on to be known as the Euromaidan protests.  In this work, we analyze open 2013 protest data  from the Center for Social and Labor Research in Ukraine.  Our analysis shows that there is self-excitation in the system even before the Euromaidan protests began and this self-excitation magnifies during the Euromaidan protests.  Our statistical analysis suggests that the government's use of force is associated with more future protests, having an inflammatory rather than suppressing effect on protests. Furthermore, we introduce Hawkes process models to understand the spatiotemporal dynamics of the protesting activity.  We find that while the protest activity spread throughout the country the dynamics of the protests were driven by Kyiv's level of activity.  Moreover, while previous work has shown that geographical proximity is a significant predictor of the spread of events, our work illustrates that the political affinity between oblasts was a far more significant factor than the geographical distance between oblasts in determining the spread of protests. This highlights the importance of social and cultural factors in shaping the dynamics of political movements. 
    
    \end{abstract_online}
    