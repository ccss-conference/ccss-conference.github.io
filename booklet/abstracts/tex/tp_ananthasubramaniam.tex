
        \begin{abstract}{Networks and Identity Drive Geographic Properties of the Diffusion of Lexical Innovation}{%
            A. Ananthasubramaniam}{%
            Michigan, MI}{%
            }
        The adoption of cultural innovation (e.g., music, beliefs, language) is often geographically correlated, with adopters largely residing within the boundaries of relatively few well-studied, socially significant areas. These cultural regions are often hypothesized to be the result of either (i) identity performance driving the adoption of cultural innovation, or (ii) homophily in the networks underlying diffusion. In this study, we show that demographic identity and network topology are both required to model the diffusion of innovation, as they play complementary roles in producing its spatial properties. We develop an agent-based model of cultural adoption, and validate geographic patterns of transmission in our model against a novel dataset of innovative words that we identify from a 10% sample of Twitter. Using our model, we are able to directly compare a  model of diffusion that combines network (homophily) and identity (performance) against simulated network-only and identity-only counterfactuals---allowing us to test the separate and combined roles of network homophily and identity performance. While social scientists often treat either network or identity as the core social structure in modeling culture change, we show that key geographic properties of diffusion actually depend on both factors as each one influences different mechanisms of diffusion. Specifically, homophily in the network principally drives spread among urban counties via weak-tie diffusion, while identity performance plays a disproportionate role in transmission among rural counties via strong-tie diffusion. Diffusion between urban and rural areas, a key component in innovation diffusing nationally, requires both network and identity. Our work suggests that models must integrate both factors in order to understand and reproduce the adoption of innovation. 
        \end{abstract}
        