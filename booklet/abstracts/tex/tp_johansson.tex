
        \begin{abstract}{Advances and Challenges for Infectious Disease Forecasting }{%
            M. Johansson}{%
            CDC, GA}{%
            }
        Infectious disease forecasting has moved from an academic exercise to reality in the past decade. Real-time probabilistic forecasts have enabled the evaluation and comparison of forecast models, leading to insights on the rapid deterioration of forecast skill as forecast horizons increase, the value of ensemble forecasts, and the importance of simple baseline models. However, many challenges have also become clearer. Comparing forecast skill to benchmarks and between models with proper scores is now routine, but how good is the current best forecast? Can it be improved? What is its value to public health? How can we build an evidence base of forecast reliability? Are forecasts improving? What are the key components for better forecasts? What is needed to extend forecast horizons while maintaining reliability? What is the value of specific datasets? How can behavior and immunity be monitored and integrated? 
        \end{abstract}
        