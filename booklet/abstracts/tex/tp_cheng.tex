
        \begin{abstract}{Multilayer networks with higher-order interaction reveal the impact of collective behavior on epidemic dynamics}{%
            C. Cheng}{%
            Binghamton, NY}{%
            }
        The ongoing COVID-19 pandemic has inflicted tremendous economic and societal losses. In the absence of pharmaceutical interventions, the population behavioral response, including situational awareness and adherence to non-pharmaceutical intervention policies, has a significant impact on contagion dynamics. Game-theoretic models have been used to reproduce the concurrent evolution of behavioral responses and disease contagion, and social networks are critical platforms on which behavior imitation between social contacts, even dispersed in distant communities, takes place. Such joint contagion dynamics has not been sufficiently explored, which poses a challenge for policies aimed at containing the infection. In this study, we present a multi-layer network model to study contagion dynamics and behavioral adaptation. It comprises two physical layers that mimic the two solitary communities, and one social layer that encapsulates the social influence of agents from these two communities. Moreover, we adopt high-order interactions in the form of simplicial complexes on the social influence layer to delineate the behavior imitation of individual agents. This model offers a novel platform to articulate the interaction between physically isolated communities and the ensuing coevolution of behavioral change and spreading dynamics. The analytical insights harnessed therefrom provide compelling guidelines on coordinated policy design to enhance the preparedness for future pandemics. 
        \end{abstract}
        