
    \begin{abstract_online}{Factors affecting variant invasion and spread in an ongoing epidemic}{%
        A. Nande}{%
        }{%
        Johns Hopkins University, Baltimore, MD}
    The COVID-19 pandemic has been characterized by variants of concern that have exhibited varying degrees of transmissibility and evasion from infection and vaccine acquired immunity. Although a lot of work has been done to characterize these phenotypes, the factors that affect selection and competition between different strains ­– which in turn govern the dynamics of their spread ­– are not well understood. In this study we use a stochastic two strain SIR model with a resident and a variant strain to analyze how the variant type (increased transmissibility or immune evasion), time of variant introduction, and the contact network structure affect the ability of a variant to invade and spread in the population. We find that apart from strains that are highly immune evasive, variants that are introduced later in an epidemic find it harder to invade. For strains that successfully invade, a variant with transmission advantage quickly increases in prevalence, and the rate at which it takes over the resident strain is faster the earlier it is introduced. Immune evasive variants on the other hand can linger at a low prevalence for a long time and increase in prevalence only after they have sufficient fitness advantage due to a build-up of immunity to the resident strain. This highlights that immune evasive variants produced during the early stages of an epidemic may not be observed until much later. We also investigate the role played by the transmission network structure, focusing on the effects of superspreading and clustering. Although inspired by the current pandemic, our approach is general and highlights some important characteristics of variant spread. Future work will consider a COVID-19 specific model to better understand the spread of Delta and Omicron, as well as to identify the types of variants that might dominate in the future. 
    
    \end{abstract_online}
    