
        \begin{abstract}{On Algorithmic interpretation of Culture}{%
            H. S. Papi}{%
            UMaine, ME}{%
            }
        Culture is regarded as any socially transmittable information and cultural evolution (CE) is the process by which cultures change and develop over time. This can occur through various mechanisms such as the transmission of ideas and practices from one generation to another, social learning, innovation, or adaptation to changing circumstances. Cultural evolution can lead to the emergence of new technologies, beliefs, values, customs, and social structures, as well as the disappearance or transformation of older ones. It is a complex and dynamic process that involves interactions between individuals, groups, and the environment. In cultural evolution domain, most of studies are focused on modeling how culture evolves for which wide range of mathematical and computational methods are used. In this presentation, we first of all are going to take an inverse approach where we want to show how cultural evolution itself is a solution-finding algorithm for various cases for which we have used 3 standard problems with different levels of difficulty. secondly, we would discuss how cultural evolution theorists can benefit from and even advance this algorithmic structure much further by drawing on the theories of culture 
        \end{abstract}
        