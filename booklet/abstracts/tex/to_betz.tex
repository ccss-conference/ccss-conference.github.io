
    \begin{abstract_online}{Multi-Population Evolutionary Game Dynamics on Networks}{%
        K. Betz}{%
        }{%
        University at Buffalo, SUNY, Buffalo, NY}
    Evolutionary game dynamics is the study of population change based on successful, or unsuccessful, strategies.  This exploration is often done on one population with a variety of possibilities to adjust the model in order to create an accurate representation of the natural world.  One method is the inclusion of strategy-dependent feedback.  We will explore environmental feedback, which is how the environment changes based on player actions.  While this has been done previously and in many differing circumstances, we instead explore the population dynamics for more than one population that can be modeled as a network.  This type of model, while originally examined in terms of cooperators or defectors, can easily be examined in terms of how a contagion spreads through a population. Using game theory, we can create environmentally dependent payoff matrices and feedback equations for each node, and the entire population, to use for the derivation of the full system.  Then, using replicator dynamics, we can create a model for the base case of two populations interacting and explore the subsequent system of differential equations to examine the possible equilibrium and their stability.  We find that the stability of the equilibrium is dependent on the interaction between the populations, which tells us that the rate of interaction can change the population equilibrium.  In terms of a contagion, the rate of interaction can change the frequency of infected in each population.  Thus, using evolutionary game dynamics, we explore the system of equations and examine equilibrium and their stability requirements to model the change of the frequency of strategies in two interacting populations. 
    
    \end{abstract_online}
    