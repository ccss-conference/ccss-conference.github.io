
        \begin{abstract}{Crowding and shape of epidemics}{%
            S. Scarpino}{%
            Northeastern, MA}{%
            }
        The temporal clustering of cases in an epidemic varies as a function of myriad socio-demographic factors. In any given location, a higher temporal concentration of cases might require a larger surge capacity in the public health system. As a consequence, developing a mechanistic theory of how variability social system organization influences the shape of epidemics holds high scientific and public health potential. Here, I discuss how social crowding impacts the shape of pathogen epidemics and connect measures of crowding to the epidemic threshold in network models. I demonstrate the utility of this model over multiple orders of magnitude in epidemic size and population complexity using high-resolution mobility and case data from the COVID-19 pandemic. Lastly, I connect variability in social system complexity to epidemic peakedness and the effectiveness of pharmaceutical interventions using mobility-driven, meta-population simulations. This work connects classical theory on social complexity from ecology to modern public health responses and raises the importance of obtaining high-resolution population and mobility data during epidemics. 
        \end{abstract}
        